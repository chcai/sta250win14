\documentclass[12pt]{article}
\usepackage{mathtools, setspace, graphicx}
\onehalfspacing
\begin{document}
\noindent Christine Cai\\
STA250 HW4\\
Winter 2014\\

The map airports.svg is a map of the (contiguous) United States, with dots representing the 30 most frequented airports in 1989. When the user mouses over an airport, a tooltip of the form ``airport code (city, state)'' is displayed. The user can also click on an airport for a plot of median arrival delays by carrier to and from said airport. This map has been tested and should be compatible with Chrome, Firefox, and Internet Explorer. Please make sure the plots directory is in the same directory as airports.svg.

The map airports.svg is made with R package maps. After plotting a map of the United States and using points() to add the airports, the plot is saved as airports.svg. The package SVGAnnotation is used to parse airports.svg and to get the plot points. Note that several other packages, such as XML, are loaded when loading SVGAnnotation. Once the nodes for plotting the airports are identified, they are passed into addToolTips() and addLink().

The map index.html in the folder airports-anim is created with R package animint. Similar to airports.svg, index.html consists of a map of the United States, with dots representing the 30 airports. However, the airports are color coded based on mean departure delays of flights leaving the given airport. Every five seconds, the carrier changes. Please refer to the legend to the right of the map; NA means there were no flights leaving the given airport of the selected carrier. The plot below the map shows the overall (across all airports) mean departure delays by carrier. Although the plot will change carrier automatically every five seconds, the user can also click on the carrier he/she wishes to see.

index.html seems to be compatible only with Firefox. Please make sure to download not just index.html, but the entire folder airports-anim.
\end{document}